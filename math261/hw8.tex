\documentclass{article}
\usepackage{geometry}
\usepackage[namelimits,sumlimits]{amsmath}
\usepackage{amssymb,amsfonts}
\usepackage{multicol}
\usepackage{mathrsfs}
\usepackage[cm]{fullpage}
\newcommand{\nc}{\newcommand}
\newcommand{\tab}{\hspace*{5em}}
\newcommand{\conj}{\overline}
\newcommand{\dd}{\partial}
\nc{\cn}{\mathbb{C}}
\nc{\rn}{\mathbb{R}}
\nc{\qn}{\mathbb{Q}}
\nc{\zn}{\mathbb{Z}}
\nc{\aff}{\mathbb{A}}
\nc{\proj}{\mathbb{P}}
\nc{\vphi}{\varphi}
\nc{\pd}[2]{\frac{\partial {#1}}{\partial {#2}}}
\nc{\ep}{\epsilon}
\nc{\topo}{\mathscr{T}}
\nc{\basis}{\mathscr{B}}
\nc{\nullset}{\varnothing}
\nc{\openm}{\begin{pmatrix}}
\nc{\closem}{\end{pmatrix}}
\DeclareMathOperator{\im}{im}
\DeclareMathOperator{\Res}{Res}
\DeclareMathOperator{\Ind}{Ind}
\nc{\ssn}[1]{\subsubsection*{#1}}
\nc{\inner}[1]{{\langle #1\rangle}}
\nc{\h}[1]{\widehat{#1}}
\begin{document}
Name: Hall Liu

Date: \today 
\vspace{1.5cm}
\subsection*{1}
Orthogonality: Let $V_1,V_2$ be irreps and take orthonormal bases to be $e_1,\ldots,e_n$ and $f_1,\ldots,f_m$. The inner product of two of these matrix-element functions $\rho_{1,ij}$ and $\rho_{2,kl}$ can be written as $\sum_{g\in G}\rho_1(g)_{ij}\rho_2(g)_{kl}=\sum_{g\in G}e_i^T\rho_1(g)e_jf_k^T\rho_2(g)f_l$. Since $e_i$ and $f_l$ don't depend on $G$, we can factor them out to get $e_i^TAf_l$, where $A$ is the sum of matrices in the middle. Now, since $A$ is $G$-invariant and linear, by Schur's lemma we must have that it's zero if the irreps are different. If the irreps are the same, then $A=\lambda_{jk}I$, where $\lambda_{jk}$ is a $G$-invariant bilinear function of $e_j$ and $f_k$. Thus, we must have some $B$ such that $\lambda_{jk}=e_j^TBf_k$, where $B$ is still $G$-invariant. Thus, $B$ is some $\mu I$, which makes the whole thing into $\mu e_i^Te_j^Tf_kf_l$ (now assuming that the $e$s and the $f$s are in the same vector space since the irreps are the same). 
\end{document}
