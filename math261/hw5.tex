\documentclass{article}
\usepackage{geometry}
\usepackage[namelimits,sumlimits]{amsmath}
\usepackage{amssymb,amsfonts}
\usepackage{multicol}
\usepackage{mathrsfs}
\usepackage[cm]{fullpage}
\newcommand{\nc}{\newcommand}
\newcommand{\tab}{\hspace*{5em}}
\newcommand{\conj}{\overline}
\newcommand{\dd}{\partial}
\nc{\cn}{\mathbb{C}}
\nc{\rn}{\mathbb{R}}
\nc{\qn}{\mathbb{Q}}
\nc{\zn}{\mathbb{Z}}
\nc{\aff}{\mathbb{A}}
\nc{\proj}{\mathbb{P}}
\nc{\pd}[2]{\frac{\partial {#1}}{\partial {#2}}}
\nc{\ep}{\epsilon}
\nc{\topo}{\mathscr{T}}
\nc{\basis}{\mathscr{B}}
\nc{\nullset}{\varnothing}
\nc{\openm}{\begin{pmatrix}}
\nc{\closem}{\end{pmatrix}}
\DeclareMathOperator{\im}{im}
\DeclareMathOperator{\Res}{Res}
\DeclareMathOperator{\Ind}{Ind}
\nc{\ssn}[1]{\subsubsection*{#1}}
\nc{\inner}[1]{{\langle #1\rangle}}
\begin{document}
Name: Hall Liu

Date: \today 
\vspace{1.5cm}
\subsection*{3.23}
By Frobenius reciprocity, for each irrep $V_i$ of $S_4$, we have $\inner{\chi_{\mathrm{Ind}(W)},\chi_{V_i}}_{S_4}=\inner{\chi_{\chi_W,\chi_{\mathrm{Res}(V_i)}}}_\inner{H}$. Thus, to decompose $\Ind(W)$ into irreps of $S_4$, we just need to take a bunch of inner products with restrictions of the irreps.
\ssn{i}
The irreps of $S_4$ restricted to $\inner{(1234)}$ are

\begin{tabular}{c|cccc}
    $\inner{(1234)}$&$1$&$(1234)$&$(13)(24)$&$(1432)$\\
    \hline
    $U$&1&1&1&1\\
    $U'$&1&$-1$&1&$-1$\\
    $V$&3&$-1$&$-1$&$-1$\\
    $V'$&3&1&$-1$&1\\
    $W$&2&0&2&0\\
\end{tabular}

The representation defined for us has character $1,i,-1,-i$ in the same order. Successively taking inner-products, we get one copy of $\Res V$ and one copy of $\Res V'$, so the induced rep is $V\oplus V'$.
\ssn{ii}
Irreps of $S_4$ restricted to $\inner{(123)}$ are

\begin{tabular}{c|ccc}
    $\inner{(123)}$&$1$&$(123)$&$(132)$\\
    \hline
    $U$&1&1&1\\
    $U'$&1&$1$&1\\
    $V$&3&0&0\\
    $V'$&3&0&0\\
    $W$&2&$-1$&$-1$\\
\end{tabular}

The representation defined has character $1,e^{2\pi i/3},e^{4\pi i/3}$ in the same order. Successively taking inner-products, we get one copy each of $\Res V$, $\Res V'$, and $\Res W$, so the induced rep is $V\oplus V'\oplus W$.
\subsection*{3.24}
For any rep $X$ of $A_5$, we have that $\Ind X=X\oplus (12)X$, where a permutation $\sigma$ from $S_5$ acts on $\Ind X$ by acting directly on $X$ if $\sigma$ is even, and by $\sigma(x+y)=\sigma(12)\cdot x+(12)\sigma\cdot y$ if $\sigma$ is odd.
\end{document}

