\documentclass{article}
\usepackage{geometry}
\usepackage[namelimits,sumlimits]{amsmath}
\usepackage{amssymb,amsfonts}
\usepackage{multicol}
\usepackage{mathrsfs}
\usepackage[cm]{fullpage}
\newcommand{\nc}{\newcommand}
\newcommand{\tab}{\hspace*{5em}}
\newcommand{\conj}{\overline}
\newcommand{\dd}{\partial}
\nc{\cn}{\mathbb{C}}
\nc{\rn}{\mathbb{R}}
\nc{\qn}{\mathbb{Q}}
\nc{\zn}{\mathbb{Z}}
\nc{\aff}{\mathbb{A}}
\nc{\proj}{\mathbb{P}}
\nc{\pd}[2]{\frac{\partial {#1}}{\partial {#2}}}
\nc{\ep}{\epsilon}
\nc{\topo}{\mathscr{T}}
\nc{\basis}{\mathscr{B}}
\nc{\nullset}{\varnothing}
\nc{\openm}{\begin{pmatrix}}
\nc{\closem}{\end{pmatrix}}
\DeclareMathOperator{\im}{im}
\DeclareMathOperator{\Res}{Res}
\DeclareMathOperator{\Ind}{Ind}
\nc{\ssn}[1]{\subsubsection*{#1}}
\nc{\inner}[1]{{\langle #1\rangle}}
\nc{\h}[1]{\widehat{#1}}
\begin{document}
Name: Hall Liu

Date: \today 
\vspace{1.5cm}
\subsection*{3.23}
By Frobenius reciprocity, for each irrep $V_i$ of $S_4$, we have $\inner{\chi_{\mathrm{Ind}(W)},\chi_{V_i}}_{S_4}=\inner{\chi_{\chi_W,\chi_{\mathrm{Res}(V_i)}}}_\inner{H}$. Thus, to decompose $\Ind(W)$ into irreps of $S_4$, we just need to take a bunch of inner products with restrictions of the irreps.
\ssn{i}
The irreps of $S_4$ restricted to $\inner{(1234)}$ are

\begin{tabular}{c|cccc}
    $\inner{(1234)}$&$1$&$(1234)$&$(13)(24)$&$(1432)$\\
    \hline
    $U$&1&1&1&1\\
    $U'$&1&$-1$&1&$-1$\\
    $V$&3&$-1$&$-1$&$-1$\\
    $V'$&3&1&$-1$&1\\
    $W$&2&0&2&0\\
\end{tabular}

The representation defined for us has character $1,i,-1,-i$ in the same order. Successively taking inner-products, we get one copy of $\Res V$ and one copy of $\Res V'$, so the induced rep is $V\oplus V'$.
\ssn{ii}
Irreps of $S_4$ restricted to $\inner{(123)}$ are

\begin{tabular}{c|ccc}
    $\inner{(123)}$&$1$&$(123)$&$(132)$\\
    \hline
    $U$&1&1&1\\
    $U'$&1&$1$&1\\
    $V$&3&0&0\\
    $V'$&3&0&0\\
    $W$&2&$-1$&$-1$\\
\end{tabular}

The representation defined has character $1,e^{2\pi i/3},e^{4\pi i/3}$ in the same order. Successively taking inner-products, we get one copy each of $\Res V$, $\Res V'$, and $\Res W$, so the induced rep is $V\oplus V'\oplus W$.
\subsection*{3.24}
For the reps $U,V,W$ of $A_5$, calculate the character of each induced irrep from (3.18). Denote the two cosets as $A_5$ and $(12)A_5$. If $g$ is even, then $g$ fixes both cosets, and so $\chi_{\Ind X}(g)=\chi_X(g)+\chi_X((12)g(12))=2\chi_X(g)$ since $(12)g(12)$ is in the same conjugacy class as $g$ (in $S_5$, but note that the three reps we're looking at have the same character on the conjugacy class of $S_5$ which splits in $A_5$). If $g$ is odd, then $g$ fixes no cosets, and the character is zero. Then, from this, we can readily compute the characters of $\Ind U$, $\Ind V$, and $\Ind W$ and find them to be what they are. 

As for $Y$ and $Z$, note that conjugating a $5$-cycle by $(12)$ takes it to the other conjugacy class in $A_5$. Thus, if $X=Y$ or $Z$ and $g$ is a $5$-cycle, we have $\chi_{\Ind X}=1$, and the formula above holds for the other conjugacy classes since they don't split. Calculating the character this way gives us $\bigwedge^2 V$.
\subsection*{3.26}
From last week's homework, we know that this group has a normal subgroup of order $7$, which means that three of its irreps are lifted from the quotient isomorphic to $\zn_3$. To figure out the conjugacy classes, first note that the cube roots of unity are $1, 2, 4$, and the normal subgroup is the set of functions $f(x)=x+\beta_0$ for $\beta_0\in[0..6]$. To see this, note that if $g=\alpha x+\beta$, then $g(f(g^{-1}(x)))=g(f((x-\beta)/\alpha))=g((x-\beta)/\alpha+\beta_0)=x+\alpha\beta_0$, which is in the subgroup. Coincidentally, this also tells us what conjugacy classes compose the normal subgroup by looking at the possibilities for $\alpha$ -- we have the identity, the one corresponding to $\beta_0=1,2,4$, and another corresponding to $\beta_0=3,5,6$.

Since we found last time that there are $5$ conjugacy classes, we only need two more. Note that if we conjugate $f(x)=\alpha_0x+\beta_0$ for $\alpha_0=2$ or $4$ by $\alpha x+\beta$, we get $\alpha_0x+(\alpha_0-1)\beta_0-\beta$. Since the non-$x$ part of this can be set to anything by varying $\beta$, we get that the last two conjugacy classes are given by $\alpha_0=2$ and $\alpha_0=4$, each with $7$ elements.

The reps of $\zn_3$ are constant on the normal subgroup and take on their usual values on the other two conjugacy classes, so we have 

\begin{tabular}{c|ccccc}
    $G$&1&$\beta_0=1,2,4$&$\beta_0=3,5,6$&$\alpha_0=2$&$\alpha_0=4$\\
    \hline
    $V_1$&1&1&1&1&1\\
    $V_2$&1&1&1&$e^{2i\pi/3}$&$e^{-2i\pi/3}$\\
    $V_3$&1&1&1&$e^{-2i\pi/3}$&$e^{2i\pi/3}$\\
    $V_4$&3\\
    $V_5$&3\\
\end{tabular}

As found on the last HW, the last two must be dimension $3$. Let's try inducing a rep from the normal subgroup $H\equiv\zn_7$, since that'd give the right dimension. Choose the rep that sends $x+1=1\in\zn_7$ to $\xi=e^{2i\pi/7}$. The cosets are the functions with $\alpha=1$, $\alpha=2$, and $\alpha=4$. Write these as $H$, $2H$, and $4H$. For the elements in $\beta_0=1,2,4$, they fix all the cosets. Conjugating $x+1$ by $x$, conjugating by $2x$ gives $x+2$, and by $4x$ gives $x+4$, so the value of the character is $\xi+\xi^2+\xi^4$. For $x+3\in\beta_0=3,5,6$, we similarly get $\xi^3+\xi^5+\xi^6$. The other conjugacy classes don't fix any cosets, so the character is zero.

Now, to show that this is an irrep, take the self-inner-product. We have $9+3(\xi+\xi^2+\xi^4)(\xi^6+\xi^5+\xi^3)+3(\xi^6+\xi^5+\xi^3)(\xi+\xi^2+\xi^4)=9+6+6=21$, so it's an irrep. Note that we can build another irrep simply by taking $x+1\mapsto e^{12i\pi/7}$ as a rep of $H$ and inducing. This gives the same character except with the values switched on the two nontrivial conjugacy classes making up $H$. We get a final character table of

\begin{tabular}{c|ccccc}
$G$&1&$\beta_0=1,2,4$&$\beta_0=3,5,6$&$\alpha_0=2$&$\alpha_0=4$\\
\hline
$V_1$&1&1&1&1&1\\
$V_2$&1&1&1&$e^{2i\pi/3}$&$e^{-2i\pi/3}$\\
$V_3$&1&1&1&$e^{-2i\pi/3}$&$e^{2i\pi/3}$\\
$V_4$&3&$(\xi+\xi^2+\xi^4)$&$(\xi^6+\xi^5+\xi^3)$&0&0\\
$V_5$&3&$(\xi^6+\xi^5+\xi^3)$&$(\xi+\xi^2+\xi^4)$&0&0\\
\end{tabular}
\subsection*{3.31}
Let the two functions be $\phi$ and $\psi$. The product of $\phi$ and $\psi$ in $\cn G$ is $\sum_{g\in G}\sum_{h\in H}\phi(g)\psi(h)e_{gh}=\sum_{k\in G}e_k\left(\sum_{gh=k}\phi(g)\psi(h)\right)$. Note that if we fix $g$, then $h=g^{-1}k$, so this becomes $\sum_{k\in G}e_k\left(\sum_{g\in G}\phi(g)\psi(g^{-1}k)\right)$, which is the element of $\cn G$ that corresponds to the convolution of $\phi$ and $\psi$.
\subsection*{3.32}
\ssn{a}
We have 
\[
    \h{\phi}(\rho)\cdot\h{\psi}(\rho)=\left(\sum_{g\in G}\phi(g)\rho(g)\right)\left(\sum_{h\in G}\psi(h)\rho(h)\right)=\sum_{k\in G}\left(\sum_{g\in G}\phi(g)\psi(g^{-1}k)\right)\rho(k)
\]
On the other hand, if we let $f=\phi*\psi$, then $f(k)=\sum_{g\in G}\phi(g)\psi(g^{-1}k)$, so $\h{f}$ is the same expression as above.
\ssn{b}
For any irrep $\rho$, we have $\rho(g^{-1})\cdot\h{\phi}(\rho)=\sum_{h\in G}\phi(h)\rho(g^{-1}h)$. Taking the trace, we have $\sum_{h\in H}\phi(h)\chi_\rho(g^{-1}h)$, or the convolution $\phi*\chi_\rho$. Then we can write the RHS as $\frac{1}{|G|}\sum_\rho\phi*(\dim(V_\rho)\chi_\rho=\frac{1}{|G|}\phi*\left(\sum_\rho\dim(V_\rho)\chi_\rho\right)$. By (2.20) and (2.19), the term in the parens is a function on $G$ that takes on the value $|G|$ at $e$ and zero elsewhere. Thus, the convolution's value on a particular element $g$ can be written as $\phi(g)|G|$, which results in $\phi(g)$ after dividing by that factor of $|G|$.
\ssn{c}
Define $\phi_g$ such that $\phi_g(g^{-1})=1$ and $\phi_g=0$ elsewhere. Then, the formula we're asked to prove reduces to the Fourier inversion formula in (b) which is known to be true. Thus, since the $\phi_g$ span $\cn G$, we just need to show that the RHS is linear in $\phi$. Since trace and summation are known to be linear, we just have to show that the hat operator is linear, which is easy beacuse the hat operator is itself a sum.
\subsection*{3}
\ssn{a}
If $f$ is a class function, then for any $g,h\in G$, $f(g)=f(hgh^{-1})$, which implies the same for any $g,h\in H$. Thus $\Res f$ is also a class function
\ssn{b}
Suppose $g,h\in G$. Then $\Ind f(hgh^{-1})=\frac{1}{|H|}\sum_{x\in G}\tilde{f}(x^{-1}hgh^{-1}x)=\frac{1}{|H|}\sum_{y\in G}\tilde{f}(y^{-1}gy)=\Ind f(g)$, where we set $y=h^{-1}x$. To show that it's a linear map, note that the extension to the big group is linear and the summation is also linear.
\subsection*{4}
Compute the character of $\Ind_{Z(G)}^GW$ for some irrep $W$ of $Z=Z(G)$. For any coset $aZ$ and any group element $g$, we have that $g$ fixes $aZ$ iff $gaZ=aZ\iff gaz_1=az_2$ for $z_1,z_2\in Z$. This is equivalent to $ga=z_1^{-1}z_2a\iff g\in Z$. Thus, group elements outside of $Z$ fix no cosets, which means that the character on them is zero. For group elements in the center, they fix all cosets, so the character is $[G:Z]\chi_W(g)$ (since $g$ can commute out of the conjugacy). Now, taking the inner product of this character with itself, we have $[G:Z]\sum_{z\in Z}|\chi_W(g)|^2=[G:Z]\inner{\chi_W,\chi_W}=[G:Z]>1$, since $G$ is non-abelian. Thus the character must be reducible.
\subsection*{5}
A representation is faithful iff its character takes on the value $\dim W$ only at the identity. Further, the value of a character is always the sum of $\dim W$ roots of unity. Then, looking at the formula for the character of the induced rep, we see that it's a sum of the values of at most $[G:H]$ characters of $W$, which corresponds to at most $[G:H]\dim W=\dim\Ind W$ roots of unity. This is equal to $\dim\Ind W$ iff all the roots of unity are $1$, but that would violate the faithfulness of $W$.
\subsection*{6}
Consider the trivial subgroup of $G$ and the trivial representation on it. Inducing this to $G$ produces the permutation rep on the cosets of the trivial subgroup, which is just the regular rep. Now, if $W$ is an irrep of $G$, then $\inner{W,V_\text{reg}}_G=\inner{W,\Ind_{\{e\}}^GV_\text{triv}}_G=\inner{Res_{\{e\}}^GW,V_\text{triv}}_{\{e\}}$. The restriction of $W$ to $\{e\}$ has the value $\dim W$, so this inner product ends up being $\dim W$, which is the multiplicity of $W$ in $V_\text{reg}$.
\end{document}

