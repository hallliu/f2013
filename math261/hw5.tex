\documentclass{article}
\usepackage{geometry}
\usepackage[namelimits,sumlimits]{amsmath}
\usepackage{amssymb,amsfonts}
\usepackage{multicol}
\usepackage{mathrsfs}
\usepackage[cm]{fullpage}
\newcommand{\nc}{\newcommand}
\newcommand{\tab}{\hspace*{5em}}
\newcommand{\conj}{\overline}
\newcommand{\dd}{\partial}
\nc{\cn}{\mathbb{C}}
\nc{\rn}{\mathbb{R}}
\nc{\qn}{\mathbb{Q}}
\nc{\zn}{\mathbb{Z}}
\nc{\aff}{\mathbb{A}}
\nc{\proj}{\mathbb{P}}
\nc{\pd}[2]{\frac{\partial {#1}}{\partial {#2}}}
\nc{\ep}{\epsilon}
\nc{\topo}{\mathscr{T}}
\nc{\basis}{\mathscr{B}}
\nc{\nullset}{\varnothing}
\nc{\openm}{\begin{pmatrix}}
\nc{\closem}{\end{pmatrix}}
\DeclareMathOperator{\im}{im}
\DeclareMathOperator{\Res}{Res}
\DeclareMathOperator{\Ind}{Ind}
\nc{\ssn}[1]{\subsubsection*{#1}}
\nc{\inner}[1]{{\langle #1\rangle}}
\begin{document}
Name: Hall Liu

Date: \today 
\vspace{1.5cm}
\subsection*{3.23}
By Frobenius reciprocity, for each irrep $V_i$ of $S_4$, we have $\inner{\chi_{\mathrm{Ind}(W)},\chi_{V_i}}_{S_4}=\inner{\chi_{\chi_W,\chi_{\mathrm{Res}(V_i)}}}_\inner{H}$. Thus, to decompose $\Ind(W)$ into irreps of $S_4$, we just need to take a bunch of inner products with restrictions of the irreps.
\ssn{i}
The irreps of $S_4$ restricted to $\inner{(1234)}$ are

\begin{tabular}{c|cccc}
    $\inner{(1234)}$&$1$&$(1234)$&$(13)(24)$&$(1432)$\\
    \hline
    $U$&1&1&1&1\\
    $U'$&1&$-1$&1&$-1$\\
    $V$&3&$-1$&$-1$&$-1$\\
    $V'$&3&1&$-1$&1\\
    $W$&2&0&2&0\\
\end{tabular}

The representation defined for us has character $1,i,-1,-i$ in the same order. Successively taking inner-products, we get one copy of $\Res V$ and one copy of $\Res V'$, so the induced rep is $V\oplus V'$.
\ssn{ii}
Irreps of $S_4$ restricted to $\inner{(123)}$ are

\begin{tabular}{c|ccc}
    $\inner{(123)}$&$1$&$(123)$&$(132)$\\
    \hline
    $U$&1&1&1\\
    $U'$&1&$1$&1\\
    $V$&3&0&0\\
    $V'$&3&0&0\\
    $W$&2&$-1$&$-1$\\
\end{tabular}

The representation defined has character $1,e^{2\pi i/3},e^{4\pi i/3}$ in the same order. Successively taking inner-products, we get one copy each of $\Res V$, $\Res V'$, and $\Res W$, so the induced rep is $V\oplus V'\oplus W$.
\subsection*{3.24}
For the reps $U,V,W$ of $A_5$, calculate the character of each induced irrep from (3.18). Denote the two cosets as $A_5$ and $(12)A_5$. If $g$ is even, then $g$ fixes both cosets, and so $\chi_{\Ind X}(g)=\chi_X(g)+\chi_X((12)g(12))=2\chi_X(g)$ since $(12)g(12)$ is in the same conjugacy class as $g$ (in $S_5$, but note that the three reps we're looking at have the same character on the conjugacy class of $S_5$ which splits in $A_5$). If $g$ is odd, then $g$ fixes no cosets, and the character is zero. Then, from this, we can readily compute the characters of $\Ind U$, $\Ind V$, and $\Ind W$ and find them to be what they are. 

As for $Y$ and $Z$, note that conjugating a $5$-cycle by $(12)$ takes it to the other conjugacy class in $A_5$. Thus, if $X=Y$ or $Z$ and $g$ is a $5$-cycle, we have $\chi_{\Ind X}=1$, and the formula above holds for the other conjugacy classes since they don't split. Calculating the character this way gives us $\bigwedge^2 V$.
\subsection*{3.26}
From last week's homework, we know that this group has a normal subgroup of order $7$, which means that three of its irreps are lifted from the quotient isomorphic to $\zn_3$. To figure out the conjugacy classes, first note that the cube roots of unity are $1, 2, 4$, and the normal subgroup is the set of functions $f(x)=x+\beta_0$ for $\beta_0\in[0..6]$. To see this, note that if $g=\alpha x+\beta$, then $g(f(g^{-1}(x)))=g(f((x-\beta)/\alpha))=g((x-\beta)/\alpha+\beta_0)=x+\alpha\beta_0$, which is in the subgroup. Coincidentally, this also tells us what conjugacy classes compose the normal subgroup by looking at the possibilities for $\alpha$ -- we have the identity, the one corresponding to $\beta_0=1,2,4$, and another corresponding to $\beta_0=3,5,6$.

Since we found last time that there are $5$ conjugacy classes, we only need two more. Note that if we conjugate $f(x)=\alpha_0x+\beta_0$ for $\alpha_0=2$ or $4$ by $\alpha x+\beta$, we get $\alpha_0x+(\alpha_0-1)\beta_0-\beta$. Since the non-$x$ part of this can be set to anything by varying $\beta$, we get that the last two conjugacy classes are given by $\alpha_0=2$ and $\alpha_0=4$, each with $7$ elements.

The reps of $\zn_3$ are constant on the normal subgroup and take on their usual values on the other two conjugacy classes, so we have 

\begin{tabular}{c|ccccc}
    $G$&1&$\beta_0=1,2,4$&$\beta_0=3,5,6$&$\alpha_0=2$&$\alpha_0=4$\\
    \hline
    $V_1$&1&1&1&1&1\\
    $V_2$&1&1&1&$e^{2i\pi/3}$&$e^{-2i\pi/3}$\\
    $V_3$&1&1&1&$e^{-2i\pi/3}$&$e^{2i\pi/3}$\\
    $V_4$&3\\
    $V_5$&3\\
\end{tabular}

As found on the last HW, the last two must be dimension $3$. Let's try inducing a rep from the normal subgroup $H\equiv\zn_7$, since that'd give the right dimension. Choose the rep that sends $x+1=1\in\zn_7$ to $\xi=e^{2i\pi/7}$. The cosets are the functions with $\alpha=1$, $\alpha=2$, and $\alpha=4$. Write these as $H$, $2H$, and $4H$. For the elements in $\beta_0=1,2,4$, they fix all the cosets. Conjugating $x+1$ by $x$, conjugating by $2x$ gives $x+2$, and by $4x$ gives $x+4$, so the value of the character is $\xi+\xi^2+\xi^4$. For $x+3\in\beta_0=3,5,6$, we similarly get $\xi^3+\xi^5+\xi^6$. The other conjugacy classes don't fix any cosets, so the character is zero.

Now, to show that this is an irrep, take the self-inner-product. We have $9+3(\xi+\xi^2+\xi^4)(\xi^6+\xi^5+\xi^3)+3(\xi^6+\xi^5+\xi^3)(\xi+\xi^2+\xi^4)=9+6+6=21$, so it's an irrep. Note that we can build another irrep simply by taking $x+1\mapsto e^{12i\pi/7}$ as a rep of $H$ and inducing. This gives the same character except with the values switched on the two nontrivial conjugacy classes making up $H$. We get a final character table of

\begin{tabular}{c|ccccc}
$G$&1&$\beta_0=1,2,4$&$\beta_0=3,5,6$&$\alpha_0=2$&$\alpha_0=4$\\
\hline
$V_1$&1&1&1&1&1\\
$V_2$&1&1&1&$e^{2i\pi/3}$&$e^{-2i\pi/3}$\\
$V_3$&1&1&1&$e^{-2i\pi/3}$&$e^{2i\pi/3}$\\
$V_4$&3&$(\xi+\xi^2+\xi^4)$&$(\xi^6+\xi^5+\xi^3)$&0&0\\
$V_5$&3&$(\xi^6+\xi^5+\xi^3)$&$(\xi+\xi^2+\xi^4)$&0&0\\
\end{tabular}

\end{document}

