\documentclass{article}
\usepackage{geometry}
\usepackage[namelimits,sumlimits]{amsmath}
\usepackage{amssymb,amsfonts}
\usepackage{multicol}
\usepackage{mathrsfs}
\usepackage[cm]{fullpage}
\newcommand{\nc}{\newcommand}
\newcommand{\tab}{\hspace*{5em}}
\newcommand{\conj}{\overline}
\newcommand{\dd}{\partial}
\nc{\cn}{\mathbb{C}}
\nc{\rn}{\mathbb{R}}
\nc{\qn}{\mathbb{Q}}
\nc{\zn}{\mathbb{Z}}
\nc{\aff}{\mathbb{A}}
\nc{\proj}{\mathbb{P}}
\nc{\pd}[2]{\frac{\partial {#1}}{\partial {#2}}}
\nc{\ep}{\epsilon}
\nc{\topo}{\mathscr{T}}
\nc{\basis}{\mathscr{B}}
\nc{\nullset}{\varnothing}
\nc{\openm}{\begin{pmatrix}}
\nc{\closem}{\end{pmatrix}}
\DeclareMathOperator{\im}{im}
\DeclareMathOperator{\Res}{Res}
\DeclareMathOperator{\Ind}{Ind}
\nc{\ssn}[1]{\subsubsection*{#1}}
\nc{\inner}[1]{{\langle #1\rangle}}
\nc{\h}[1]{\widehat{#1}}
\begin{document}
Name: Hall Liu

Date: \today 
\vspace{1.5cm}

\subsection*{4.6}
Take the corresponding Young diagram. There are $n$ tabloids, each identified by which number goes in the bottom box. Index these as $w_i$, $i\in[1..n]$. For each of these, choose the representative which puts the rows in increasing order. Then, for $i\neq1$, the subgroup that preserves the columns of $w_i$ is the one generated by $(1\ i)$, and for $i=1$ it's the one generated by $(1\ 2)$. This is because any other permutation would move the columns with only one box around. We then have $\kappa_i=e-(1\ i)$ and $\kappa_1=e-(1\ 2)$, so $e_i=w_i-w_1$ and $e_1=w_1-w_2$. Note that $S_n$ acts on the $w_i$ as the permutation rep on $n$ elements, so the span of the $e_i$ is a subrepresentation of the permutation rep. We know that this only has one subrep of the correct dimension, and it's the standard rep.
\subsection*{4.14}
Induct by removing columns from the left of the Young diagram. Assume that the result holds for all $k<n$. First, let's do the general case: assume that the tableau we have is not one of the special cases, and assume that removing a column will not get us to one of the special cases. Later, we'll show the base cases: if we have a special case and add on a column, it will produce something that satisfies the hypothesis. Note that adding a column to the left of any special case will generate a non-special case.

Let the partition be $n=\lambda_1+\cdots+\lambda_r$. Then, the first column will have $r$ boxes, and removing those will generate a Young diagram on $n-r$ boxes. Label the hook-lengths of each box in the diagram as $a_{ij}$. By the inductive hypothesis, we have that $\frac{(n-r)!}{\prod_{j>1}a_{ij}}>n-r$.  Dividing by $n-r$, we have $\frac{(n-r-1)!}{\prod_{j>1}a_{ij}}>1$, so now we want to show that $\frac{n(n-1)\cdots(n-r)}{\prod a_{i1}}>n$, because then multiplying the two things together gets us our desired result.

Compute the $a_{i1}$ explicitly. The $i1$th box has $r-i$ boxes under it (not including itself) and $\lambda_i$ boxes to the right of it (including itself), so the hook length is $r-i+\lambda_i$. For $i=1$, this quantity is less than $n-1$. This is because the hook length of the upper-left box is $n$ iff the diagram is a big hook, and removing the first column off that results in a special case. Also the hook length is $n-1$ iff the diagram is a big hook with an extra box in the crook, and removing the first column of that gives the standard rep special case. Now, note that $a_{i1}\geq a_{(i+1)1}+1$, since $\lambda_i$ is nonincreasing. Thus, we have that $a_{i1}<n-i$, so the product of them is less than $(n-1)(n-2)\cdots(n-r)$. This gives us our desired result.

Now we need to handle all the special cases. Adding on a column of length $r$ to the Young diagram of the trivial rep on $S_k$ forms a big hook corresponding to the partition $(k+1)+1+\cdots+1$. The hook length of the upper element is $r+k$. The product of the hook lengths on the upper arm is $k!$. The product of the hook lengths on the lower arm is $(r-1)!$. Thus, the dimension of the rep is $\frac{(r+k)!}{(r+k)k!(r-1)!}=\frac{(r-1+k)!}{k!(r-1)!}=\binom{r-1+k}{r-1}$. This is larger than $r-1+k$ as long as $r-1\geq2$, which isn't a problem because tacking on a column of length $2$ gives the diagram corresponding to the standard rep.

I'm doing the rest of this problem by hand on paper. See attached. 
\subsection*{3}
View each of the lengths as a bin of size $\lambda_i$ into which numbers can be put into. Order these bins by decreasing size, and put the numbers from $1$ to $n$ into them in any order. There are $n!$ ways to do this. However, since the individual cycles are invariant under cyclic permutation of their entries, we overcount by a factor of $\lambda_i$ for the $i$th bin, so we need to divide those out. Further, we need to account for repeats in the cycle sizes -- these aren't eliminated by ordering things in decreasing order. Thus, if a certain cycle length is repeated $r$ times, we need to divide out by $r!$ for each instance of repeated cycle size. If we write $n=a_1\lambda_1+\cdots+a_r\lambda_r$ (accounting for the repeats), then we have that the size of the conjugacy class is $\frac{n!}{\prod\lambda_i^{a_i}\cdot\prod a_i!}$.
\subsection*{4}
We know what the trivial, alternating, and standard reps look like from what we've done already. The partition $2+1+1+1$ corresponds to an irrep of dimension 4 (by hook-length) that's not the standard rep, so it must be $V'$. The partition $3+1+1$ corresponds to an irrep of dimension 6, and the only one of those is $\bigwedge^2 V$. The partitions $3+2$ and $2+2+1$ each have dimension 5, so one of them is $W$ and the other is $W'$. To distinguish between them, evaluate the character of one of them on the conjugacy class of $(1\ 2)(3\ 4\ 5)$. Fortunately, the book did this for us for $3+2$, and that comes out to $1$, which corresponds to $W$. Thus, $2+2+1$ must correspond to $W'$.
\subsection*{5}
Let the basis vector corresponding to $i,j$ in the bottom row be $w_{ij}$, and in general index everything by $i,j$.
Note that the irrep corresponding to this partition is always contained in $M_D$: this is because the $e_T$ are linear combinations of the $w_T$. $M_D$ has dimension $\binom{n}{2}=\frac{n(n-1)}{2}$, and by the hook length formula, the irrep corresponding to $D$ has dimension $\frac{n(n-3)}{2}$. Thus, we have $n$ dimensions left over to fill.

Let $a_k=\sum_{i\neq k} w_{ik}$ for $k\in[1..n]$. For any $\sigma$, we have 
\[\sigma\cdot a_k=\sum_{i\neq k} w_{\sigma(i)\sigma(k)}=\sum_{\sigma(i)\neq\sigma(k)} w_{\sigma(i)\sigma(k)}=\sum_{i\neq\sigma(k)}w_{i\sigma(k)}=a_{\sigma(k)}\]
so we have that the span of the $a_i$ is a subrep of the rep we're looking at. We want to show that the $a_i$ are independent: take some linear combination of the $a_i$, $\sum c_ia_i$. Then, expanding, we have $\sum_i c_ia_i=\sum_i c_i\sum_{j\neq i}w_{ji}$. Note that each $w_{ij}$ carries a coefficient of $c_i+c_j$ in this sum, so in order for the linear combination to equal zero, we must have $c_i+c_j=0$ for all $i,j$. This implies that $c_i=0$ for all $i$ unless $n=2$, in which case we just get the trivial rep. Thus, the span of the $a_i$ is isomorphic to the permutation rep with dimension $n$, which itself decomposes into a copy of the standard rep and a copy of the trivial rep.

\end{document}
