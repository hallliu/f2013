\documentclass{article}
\usepackage{geometry}
\usepackage[namelimits,sumlimits]{amsmath}
\usepackage{amssymb,amsfonts}
\usepackage{multicol}
\usepackage{mathrsfs}
\usepackage[cm]{fullpage}
\newcommand{\nc}{\newcommand}
\newcommand{\tab}{\hspace*{5em}}
\newcommand{\conj}{\overline}
\newcommand{\dd}{\partial}
\nc{\cn}{\mathbb{C}}
\nc{\rn}{\mathbb{R}}
\nc{\qn}{\mathbb{Q}}
\nc{\zn}{\mathbb{Z}}
\nc{\ff}{\mathbb{F}}
\nc{\fq}{\mathbb{F}_q}
\nc{\fqsq}{\mathbb{F}_{q^2}}
\nc{\aff}{\mathbb{A}}
\nc{\proj}{\mathbb{P}}
\nc{\vphi}{\varphi}
\nc{\pd}[2]{\frac{\partial {#1}}{\partial {#2}}}
\nc{\ep}{\epsilon}
\nc{\topo}{\mathscr{T}}
\nc{\basis}{\mathscr{B}}
\nc{\nullset}{\varnothing}
\nc{\openm}{\begin{pmatrix}}
\nc{\closem}{\end{pmatrix}}
\DeclareMathOperator{\im}{im}
\DeclareMathOperator{\Res}{Res}
\DeclareMathOperator{\Ind}{Ind}
\nc{\ssn}[1]{\subsubsection*{#1}}
\nc{\inner}[1]{{\langle #1\rangle}}
\nc{\h}[1]{\widehat{#1}}
\begin{document}
Name: Hall Liu

Date: \today 
\subsection*{5.7}
First, the conjugacy classes of $PGL(2,q)$. The conjugacy classes originating from the scalar matrices $\openm x&0\\0&x\closem$ all collapse down to one class, the one containing only the identity. The conjugacy classes containing matrices which have a Jordan block of size $2$ also all collapse down to one class: that containing $\openm 1&1\\0&1\closem$, and its size is still $q^2-1$. If the original matrix has two distinct eigenvalues over $\fq$, then by moving into $PGL$, we can set one of those eigenvalues to $1$ and the other to any of $q-1$ choices, making for $(q-1)/2$ conjugacy classes (accounting for order). The size of thse conjugacy classes stays the same ($q(q+1)$) unless the other eigenvalue is $-1$, in which multiplying by the scalar matrix $\openm -1&0\\0&-1\closem$ takes an element of the conjugacy class to another element of the conjugacy class, which halves the size of the conjugacy class in $PGL$. Similarly, if the original matrix has two distinct eigenvalues over $\fq(\sqrt{\ep})$, then we can apply a scalar matrix so that the $\sqrt{\ep}$-part of the eigenvalues is $1$. The same issue with negatives arises when the $\fq$-part is zero, making for $(q-1)/2$ classes with $q(q-1)$ elements and one class with $q(q-1)/2$ elements. 

Now, to find the characters, we look at characters of $GL(2,q)$ that take on the same value mod a scalar matrix. In the $U_\alpha$ type, the only such ones are the trivial rep and the alternating rep, since for all other choices of $\alpha$, $U_\alpha$ will vary on the scalar matrices themselves. The same holds for the $V_\alpha$ type, for $\alpha(g)=1$ and $\alpha(g)=-1$, where $g$ is a generator.

For the $W_{\alpha,\beta}$ type, we need for $\alpha(x)\beta(x)$ to be constant no matter what the value of $x$ is. This works if $\alpha$ is the inverse of $\beta$ (in the sense of pointwise multiplication in $\cn$), so we have $(q-1)/2$ choices here. However, this won't work if $\alpha$ is the alternating rep because it's its own inverse, so we take away one and get $(q-3)/2$ choices.

For the $X_\vphi$ type, we need $\vphi(x)$ to be constant, which corresponds to the reps of $\fqsq^\times$ that remain constant on $\fq^\times$. There are $q(q-1)/2$ choices for $\vphi$ to begin with, and adding this restriction divides the number by $q$, so we get $(q-1)/2$ irreps of this type. 

%A table summarizing this stuff follows:
%
%\begin{tabular}{c|c|c|c|c|c|c|c}
%    &1&$q^2-1$&$q(q+1)$&$q(q+1)/2$&$q(q-1)$&$q(q-1)/2$\\
%    &Identity&$\openm 1&1\\0&1\closem$&$\openm1&0\\0&x\closem$&$\openm1&0\\0&-1\closem$&$\openm x&\ep\\1&x\closem$&$\openm0&\ep\\1&0\closem$\\
%    \hline
%    $U_+$&1&1&1&1&1&1&1\\
%    $U_-$&1&1&$\alpha(y)$&$\alpha(-1)$&$
%         
%\end{tabular}
\subsection*{5.8}
In general, $\proj^1(\fq)$ has $q+1$ points. The action of $SL(2,2)$ on $\proj^1(\ff_2)$ gives a map to $S_3$ which will be an isomorphism if it's injective, since $SL(2,2)$ also has six elements. Manually checking the elements, none of them act trivially, so we have an isomorphism.

Look at $PSL(2,3)$. We know that $SL(2,3)$ has order $3(3^2-1)=24$, so quotienting by the center (with 2 elements) will just produce a group of order $12$. Since $\proj^1(\ff_3)$ is invariant under scalar multiplication anyway, we can define the action by matrix multiplication. Once again, this produces a map from $PSL(2,3)$ to $S_4$. We know that it's injective, since a matrix acting trivially on $\proj^1(\ff_3)$ means that it's a scalar multiple of the identity, which doesn't happen because we modded out by all the scalar matrices. Thus, since $A_4$ is the only subgroup of $S_4$ of order $12$, $PSL(2,3)$ is isomorphic to $A_4$.

$SL(2,4)$ has order $4(4^2-1)=60$ and it acts on $\proj^1(\ff_4)$ (5 elements) by matrix multiplication. Here, the only scalar matrix producing determinant $1$ in the group is the identity, so the map's still injective. $A_5$ is the only subgroup of $S_5$ of order $60$, so there we have it.
\subsection*{5.9}
From looking at the character table for $GL(2,q)$, we see that the magnitude of all the character values outside of the scalar matrices is bounded above by $2$, since the functions $\alpha,\beta$, and $\vphi$ all take values on the unit circle. Further, they're all of dimension $>2$ for the $q$s we're concerned with. Thus, since all but $4$ of the irreps of $SL(2,q)$ arise from restrictions of the irreps of $GL(2,q)$, those $q$ irreps of $SL(2,q)$ cannot have any normal subgroup between $\{\pm1\}$ and $SL(2,q)$ as its kernel.

As for the other four, the discussion in the book gives their dimensions as $(q\pm1)/2$. The value of their characters is also half of the value of the characters for the irreps with twice the dimension on all but $4$ conjugacy classes, so that bound is still good. Finally, the book gives values for the characters on the last $4$ conjugacy classes as being bounded somewhere in the ballpark of $\frac{1}{2}\sqrt{q}$, so that'll be less than $(q\pm1)/2$ too, at least for large $q$. We can check manually that it holds for a few low values, so that's good too.
\subsection*{5.11}
Pretty sure there's something wrong with this problem -- nontrivial permutation reps are never irreducible because they contain a copy of the trivial rep, and $GL(3,q)$ definitely doesn't act trivially on the projective plane.
\subsection*{3}
%The value of $\chi_V(A)$ for some $A\in GL(2,q)$ is the number of distinct eigenvectors that $A$ has over $\fq$, since each eigenvector corresponds to a line in $\fq^2$ that $A$ fixes. We can use this to calculate the character of $V$. 
%
%$\proj^1(\fq)$ has $q+1$ points, and that's the value of the character on all of the scalar matrix conjugacy classes, as all they do is multiply by a scalar, which corresponds to no action at all.
%
%For the classes with representative $\openm x&1\\0&x\closem$, they have two eigenvectors, $[-x/(1-x):1]$ and $[1:0]$ when $x\neq1$, and only $[1:0]$ when $x=1$.
%
%For the classes with representative $\openm x&0\\0&y\closem$ with $x\neq y$, they have two eigenvectors, $[0:1]$ and $[1:0]$.
%
%For the classes with representative $\openm x&\ep y\\y&x\closem$ with $y\neq0$, they have no eigenvectors because all their eigenvalues are not in $\fq$.
%
%Thus, we have the character of this thing. Taking the inner product with itself, we get the numerator as $(q-1)\cdot(q+1)^2+(q^2-1)\cdot4\cdot(q-2)+(q^2-1)+(q^2+q)\cdot4\cdot\frac{(q-1)(q-2)}{2}=(q^2-1)(q^2+3q-6)$. If we divide by the order of the group, we get 
This is in the book.
\subsection*{4}
\ssn{a}
For any matrix in $GL(2,q)$, we can write it as $sIB$, where $B\in SL(2,q)$ and $s=\det(A)$ by taking $B=\frac{1}{s}A$. This shows that $GL(2,q)=Z\cdot SL(2,q)$. It is also evident that the intersection of the two subgroups is trivial, since the only matrix in $Z$ having unit determinant is the identity. $Z$ is normal in $GL(2,q)$ since it's the center, and $SL(2,q)$ is a normal subgroup because conjugating preserves the determinant. Thus we have all the criteria required for a group to be the internal direct product of two subgroups.
\ssn{b}
For the conjugacy classes, online sources tell me that the conjugacy classes don't split when restricting if $q$ is even. Thus, the scalar matrices represent one conjugacy class, the identity (since $-1=1$). Similarly, the size-2-Jordan-block matrices go to one class, the one containing $\openm 1&1\\0&1\closem$ still with $q^2-1$ elements. The matrices with two distinct eigenvalues must have their eigenvalues as $\lambda$ and $\lambda^{-1}$, but $\lambda$ can't be $0$ or $1$, so there are $(q-2)/2$ of these classes (dividing by two to account for switching them). These classes have size $q(q+1)$. Finally, for the matrices of the form $\openm x&\ep y\\y&x\closem$, we must have $\|x+y\sqrt{\ep}\|=1$, and there are $q/2$ of these up to conjugacy (note: $\ep$ might be something funky if $q$ is even, since there aren't any non-square elements. I'm just going to hope that this doesn't break anything). This gives a total of $q+1$ conjugacy classes. The beginning of the char table thus looks like

\begin{tabular}{c|c|c|c|c}
    &1&$q^2-1$&$q(q+1)$&$q(q-1)$\\
    &Id&$\openm 1&1\\0&1\closem$&$\openm\lambda&0\\0&\lambda^{-1}\closem$($(q-2)/2$ of these)&$\openm x&\ep y\\y&x\closem$($q/2$ of these)\\
    \hline
\end{tabular}

If we restrict the $U_\alpha$ irreps, we get the trivial rep. $q$ more to go.

If we restrict the $V_\alpha$ irreps, we get an irrep with character $q, 0, 1, -1$. Working out the inner product of this with itself gives $1$, so this is another irrep. $q-1$ more to go.

If we restrict the $W_{\alpha, 1}$ irreps, we get irreps if $\alpha^2\neq 1$ (but $\alpha^2=1$ implies $\alpha=1$ b/c $q$ even, so that case goes away). The character is $q+1, 1, \alpha(x)+\alpha(x^{-1}), 0$. There are $q-1$ irreps $\alpha$ of $\fq^\times$, but we can't have $\alpha$ be the identity, so there are $q-2$ choices. In addition, picking $\beta=\alpha^{-1}$ results in the same irrep, so we actually have $\frac{q-2}{2}$ irreps of this type.

If we restrict the $X_\vphi$ irreps, following the discussion in the book gives $q/2$ irreps in $SL(2,q)$, since there are no $\vphi$s such that $\vphi^2=1$.

Adding these together, we get $q+1$ reps, so we're done.
\ssn{c}
If we look at all the non-identity conjugacy classes and the non-trivial irreps, the magnitude of all the characters is bounded above by $2$. Since if $q>2$ the dimensions are $\geq3$, we can't have any normal subgroups, which means $SL(2,q)$ is simple.
\end{document}
