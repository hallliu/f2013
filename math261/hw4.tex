\documentclass{article}
\usepackage{geometry}
\usepackage[namelimits,sumlimits]{amsmath}
\usepackage{amssymb,amsfonts}
\usepackage{multicol}
\usepackage{mathrsfs}
\usepackage[cm]{fullpage}
\newcommand{\nc}{\newcommand}
\newcommand{\tab}{\hspace*{5em}}
\newcommand{\conj}{\overline}
\newcommand{\dd}{\partial}
\nc{\cn}{\mathbb{C}}
\nc{\rn}{\mathbb{R}}
\nc{\qn}{\mathbb{Q}}
\nc{\zn}{\mathbb{Z}}
\nc{\aff}{\mathbb{A}}
\nc{\proj}{\mathbb{P}}
\nc{\pd}[2]{\frac{\partial {#1}}{\partial {#2}}}
\nc{\ep}{\epsilon}
\nc{\topo}{\mathscr{T}}
\nc{\basis}{\mathscr{B}}
\nc{\nullset}{\varnothing}
\nc{\openm}{\begin{pmatrix}}
\nc{\closem}{\end{pmatrix}}
\DeclareMathOperator{\im}{im}
\DeclareMathOperator{\sym}{Sym}
\nc{\ssn}[1]{\subsubsection*{#1}}
\begin{document}
Name: Hall Liu

Date: \today 
\vspace{1.5cm}
Taking inner products of characters is a pain, so I used the following code to reduce the amount of petty arithmetic I had to do.
\begin{verbatim}
def innerprod(a,b,weights):
    assert len(a) == len(b)
    assert len(b) == len(weights)

    s = 0
    for i in range(len(b)):
        s += a[i].conjugate() * b[i] * weights[i]

    return s
\end{verbatim}

\subsection*{3.1}
We know that the permutation representation of $S_5$ on $\cn^5$ has character $\openm 5&3&2&1&0&1&0\closem$ (listed in the same order as in the book) just by counting the number of elements fixed by each conj. class. We also know that this rep is the direct sum of the standard rep and a copy of the trivial rep, so subtracting out the character of the trivial rep from this one should give the character of standard rep, or $\openm4&2&1&0&-1&0&-1\closem$.

The alternating rep has character $\openm1&-1&1&-1&1&1&-1\closem$. Tensoring this together with the standard rep, we multiply their characters and get $\openm4&-2&1&0&-1&0&1\closem$. We can verify that this is an irrep by taking the inner product with itself, and it indeed comes out to $1$ (calculations not shown because pointless typing)
\subsection*{3.2}
\ssn{a}
From the previous HW, the character of $\sym^2(V)$ is $\frac{\chi_V(g)^2+\chi_V(g^2)}{2}$. Calculating this for the rep $V$ of $S_5$, we have that the character is $\openm10&4&1&0&0&2&1\closem$. 
\ssn{b}
Taking the inner product of this character with itself gives $3$, so by looking at the expansion of the inner product of the irreducible decomposition of $\chi_{\sym^2(V)}$ with itself, we see that there must be three irreducibles in there to give an inner product of $3$.
\ssn{c}
Successively taking inner products of $\sym^2(V)$ with the five irreps we already know about, we find that we get an inner product of $1$ with $U$ and $V$. Subtracting these two out of the character of $\sym^2(V)$, we get $\openm5&1&-1&-1&0&1&1\closem$, which we know must be an irrep due to (b). Then, we can complete the table using the fact that the remaining irrep must have dimension $5$ and the orthogonality of the columns of the table.
\subsection*{3.3}
Character of $\bigwedge^2(W)$ is $\openm10&-2&1&0&0&-2&1\closem$. Taking inner products of this with all the irreps, we see that it takes the value $1$ at $V'$ and $\bigwedge^2V$, so that's the decomposition.

Character of $\sym^2(W)$ is $\openm15&3&0&1&0&3&0\closem$. Taking inner products with the irreps, we get the value $1$ at $U$, $V$, $W$, and $W'$.

Character of $V\otimes W$ is $\openm20&-2&-1&0&0&0&1\closem$. This decomposes into $V'$, $\bigwedge^2 V$, $W$, and $W'$.
\subsection*{3}
\ssn{a}
Jordan$\implies$: Suppose $H$ is normal. Then, $H$ must be a union of conjugacy classes, so if it contains one element from each conjugacy class, it must contain all elements in $G$. Thus, we can assume $H$ is not normal, and let $G$ act on the set of subgroups conjugate to $H$ (let's call it $A$) by conjugation. Then, by Jordan's theorem, there exists some $g\in G$ such that $g$ fixes no subgroups in $A$. Consider the conjugacy class of this $g$. Every element in this conjugacy class fixes no subgroups in $A$, for if $g'=fgf^{-1}$ fixes some $a\in A$, then $g\cdot (f^{-1}\cdot a)=f^{-1}\cdot (g'\cdot a)=f^{-1}\cdot a$. Now consider the normalizer of $H$. This is the set of all elements in $G$ which fix $H$ on conjugation. Clearly it is disjoint with the conjugacy class of $g$, but since the normalizer of $H$ contains $H$, $H$ is also disjoint with the conjugacy class of $g$.

$\implies$Jordan: Let $G$ act transitively on $X$. Let $|X|=n$, and let $H_1,\ldots,H_n$ be the stabilizer subgroups of the the elements of $X$. None of the $H_i$ are equal to $G$, for that would mean that the action is not transitive. For any pair $i,j<n$, let $h_i\in H_i$ arbitrarily. There exists some $g\in G$ such that $g\cdot i=j$, so $gh_ig^{-1}\cdot j=gh_i\cdot i=g\cdot i=j$, which means that $H_i$ is conjugate to $H_j$ for all pairs $i,j$. By the assumption, there exists some conjugacy class of $G$ disjoint from $H_1$, and now we know that it's disjoint from all the $H_i$. This means that no elements in this conjugacy class fix any elements of $X$.
\ssn{b}
(b)$\implies$)(a): If there are two distinct characters equal on $H$, then $H$ cannot intersect all the conjugacy classes of $G$, for that would make any two characters equal on $H$, on $G$.

(a)$\implies$(b): Consider the set of characters on $G$ as a $\zn$-module. Then, the set of irreducible characters forms a basis for this $\zn$-module. Now consider the set of characters on $H$ which satisfy the property that if two conjugacy classes in $H$ result from a conjugacy class of $G$ splitting when going into the subgroup, then the character takes the same value on both classes. This is a $\zn$-module as well, and we have a nice map from the characters on $G$ to this module just by restriction to the subgroup. Our aim, then, is to show that this map is not injective. To do this, note that the maximum size of a basis of the second module is at most the number of conjugacy class in $G$, but since $H$ is disjoint from at least one conjugacy class of $G$, this is actually a strict inequality. Thus, the map is not injective, so there are two distinct characters on $G$ which restrict to the same character on $H$.
\subsection*{4}
\ssn{a}
Suppose $G/N$ is abelian. Then, for any $g,h\in G$, we have $ghN=hgN$, or $g^{-1}h^{-1}ghN=N$, or $g^{-1}h^{-1}gh\in N$, which implies that all elements of the commutator subgroup are in $N$.
\ssn{b}
We know that every finite abelian group has as many irreps as elements, all of dimension $1$ (since each element forms its own conjugacy class), so we want to show that each dimension $1$ irrep of $G$ is in fact an irrep of $G/G'$ lifted to $G$. Consider any one-dimensional rep of $G$ labeled as $\rho$. Since $\cn^\times$ is commutative, we have $G/\ker(\rho)=\im(rho)$ is abelian, so by (a), $\ker(\rho)$ contains $G'$. Thus, the irrep of $G$ is actually an irrep of $G/G'$.
\subsection*{5}
%First, note that if the trivial rep is the only irrep of degree $1$, then we have that $G'=G$ and thus $G$ is not solvable unless $|G|=1$. Otherwise, we want to determine the number of degree 1 irreps of $G'$, so we can continue this process recursively until either $G$ becomes abelian or the earlier condition is reached. Note that if we have $N\triangleleft G$ such that $G/N$ is abelian, then $N$ is the kernel of some intersection of the degree 1 reps of $G$. Thus, since $G'$ is the minimal such normal subgroup, $G'$ is actually the intersection of all the kernels of the degree 1 reps of $G$. 

%Now, to actually find the degree 1 irreps of $G'$, we can use Clifford's theorem, which in this case gives us that 
Using the fact that every normal subgroup is the intersection of the kernels of some collection of irreps, we can find every normal subgroup of $G$ by taking all the intersections of kernels of the various irreps. This in fact results in all the inclusion relations between the subgroups as well, since the normal subgroups found are expressed in terms of the union of conjugacy classes of $G$.

Now, use the definition of solvability given by the existence of a chain $\{1\}\triangleleft G_1\triangleleft\cdots\triangleleft G_{n-1}\triangleleft G$ where all the quotients are simple and abelian (i.e. of prime order). If we can show that all the $G_i$ are in fact normal subgroups of $G$, then we can just walk through the normal subgroups found from the character table and look for such a chain. 

Proceed by induction. Suppose that in the above chain, $G_i$ is normal in $G$. We want to show that $G_{i-1}$ is normal in $G$. We know that $G_{i-1}$ is actually a maximal normal subgroup of $G_i$ due to the fourth isomorphism theorem. In addition, $G_{i-1}$ is solvable, which makes it a maximal solvable normal subgroup. However, this is just the solvable radical of $G_i$, making it unique. Now, conjugating $G_{i-1}$ by any element of $G$ will produce another solvable normal subgroup of $G_i$, so it follows that $G_{i-1}$ is invariant under conjugation by $G$, so $G_{i-1}\triangleleft G$.
\subsection*{6}
\ssn{a}
By Sylow's theorems, $G$ must have a subgroup of order $13$. The number of these Sylow-$13$ subgroups must divide $3$ and be equal to $1$ mod $13$, so there must only be one, which means it's a normal subgroup of order $13$. Then, we have $G/\zn_{13}\equiv\zn_3$, which means that $G$ has three one-dimensional irreps which are lifts of the three irreps of $\zn^3$. By (4a), $\zn_{13}$ as a subgroup of $G$ must contain $G'$ since the quotient is abelian, but since $\zn_{13}$ has no proper subgroups, it must be the commutator of $G$. Thus, the three $1$d irreps are the only $1$d reps. Now, since we know that the sum of squares of the degrees of irreps is the order of the group, the sum of the squares of the degrees of the remaining irreps must come out to $36$, with them all having degree at least $2$. By that theorem from class, the degrees of the remaining irreps must divide $39$, so it follows that there must be $4$ additional irreps of degree $3$.
\ssn{b}
Three $1$d irreps and four $3$d irreps makes for $7$, which is also the number of conjugacy classes.
\ssn{c}
Let $G$ now have order $21$. There must be a normal subgroup of order $7$ by the same argument as above. Taking the quotient by this subgroup results in $\zn_3$ again, which means that there are three $1$d irreps lifted up. This order $7$ subgroup is also the commutator since it's of prime order, so the three $1$d irreps are the only $1$d irreps. Then using the above argument again, there must be two remaining irreps of degree $3$, making for a total of $5$ irreps and thus $5$ conjugacy classes.


\end{document}
