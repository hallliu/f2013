\documentclass{article}
\usepackage{geometry}
\usepackage[namelimits,sumlimits]{amsmath}
\usepackage{amssymb,amsfonts}
\usepackage{multicol}
\usepackage{graphicx}
\usepackage{mathrsfs}
\usepackage[cm]{fullpage}
\newcommand{\tab}{\hspace*{5em}}
\newcommand{\conj}{\overline}
\newcommand{\dd}{\partial}
\newcommand{\ep}{\epsilon}
\newcommand{\openm}{\begin{pmatrix}}
\newcommand{\closem}{\end{pmatrix}}
\DeclareMathOperator{\cov}{cov}
\DeclareMathOperator{\var}{var}
\DeclareMathOperator{\rk}{rk}
\DeclareMathOperator{\im}{im}
\newcommand{\nc}{\newcommand}
\newcommand{\rn}{\mathbb{R}}
\nc{\h}[1]{\widehat{#1}}
\nc{\ssn}[1]{\subsubsection*{#1}}
\nc{\inner}[2]{\langle #1,#2\rangle}
\begin{document}
Name: Hall Liu

Date: \today 
\vspace{1.5cm}
\subsection*{1}
\ssn{a}
Taking a plot of residuals versus index, we have the following:

\includegraphics[width=0.6\textwidth]{final_files/1a_resids.png}

There doesn't seem to be any nonconstant variance, though there might be a few runs here and there and a point pretty far off from the rest near the middle. In addition, nonlinearity might also be a possiblility, as there's a noticable bend in the residuals.

Since we're dealing with temporal data, it's a good idea to check for correlated errors by plotting successive residuals against each other. We obtain the following plot:

\includegraphics[width=0.6\textwidth]{final_files/1a_succ.png}

There seems to be a slight linear trend. To check this numerically, we compute the Durbin-Watson statistic and get a value of $1.2472$ with a $p$-value of $4.687\times10^{-5}$, indicating that there's probably autocorrelation.
\ssn{b}
Computing a generalized least squares model, we have the estimate of $\rho$ as $0.4003$ with confidence interval $(0.1987, 0.5694)$.
\ssn{c}
If we use the estimate for $\rho$ from the \verb|gls| fit, we can obtain a covariance matrix $\Sigma=SS^T$ and transform the residuals by taking $S^{-1}e$, which provides us with an RSS and therefore an estimate of $\sigma^2$. Then, we can fit a smaller model with gls, fixing the correlation at the estimated $\rho$, then transform the residuals from that accordingly and perform the $F$-test as usual.

This is the R code for testing the quadratic model against the linear model. The others are all similar to this, but with different numbers. The \verb|findRSS| function is attached in the appendix.

\begin{verbatim}
years = 1:100
model2 = gls(Nile ~ years + I(years^2), method="ML", correlation=corAR1(form=~years))
rho = coef(model2$modelStruct$corStruct, unconstrained=FALSE)
syy = findRSS(rho, gls(Nile ~ years, method="ML", correlation=corAR1(value=rho, form=~years, fixed=TRUE)))
rss = findRSS(rho, model2)
ssreg = syy-rss
msreg = ssreg / (model2$dims$p - 1)
f = msreg / (rss / (100-model2$dims$p))
p = pf(f, 1, 97)
\end{verbatim}

Testing the successive models against each other, we have that the $F$ value for linear against intercept is $41.034$, for quadratic against linear is $11.986$, and cubic against quadratic is $0.0232$. Since addition of the cubic terms gives no additional significant improvement to the model, we conclude that the quadratic model is the best choice.
\end{document}
