\documentclass{article}
\usepackage{geometry}
\usepackage[namelimits,sumlimits]{amsmath}
\usepackage{amssymb,amsfonts}
\usepackage{multicol}
\usepackage{graphicx}
\usepackage[cm]{fullpage}
\newcommand{\tab}{\hspace*{5em}}
\newcommand{\conj}{\overline}
\newcommand{\dd}{\partial}
\newcommand{\ep}{\epsilon}
\newcommand{\openm}{\begin{pmatrix}}
\newcommand{\closem}{\end{pmatrix}}
\DeclareMathOperator{\cov}{cov}
\newcommand{\nc}{\newcommand}
\newcommand{\rn}{\mathbb{R}}
\nc{\ssn}[1]{\subsubsection*{#1}}
\begin{document}
Name: Hall Liu

Date: \today 
\vspace{1.5cm}

\subsection*{2.1}
\ssn{1}
<include picture here>

Based on this plot, a simple linear regression model may not be the best choice -- there are two plausible ways to view the trend. Either we can treat the two points at $(165.3,77.8)$ and $(169.6,71.2)$ as outliers and fit the rest of the data to a linear model, or we can decide that the two points that look like outliers are actually the result of random fluctuation and include them in the fit as well.
\ssn{2}
Let the $x$ vector denote the heights, the $y$ vector denote the weights. We have $\sum x_i=1655.2$ and there are $10$ data points, so we have $\conj{x}=165.52$. Similarly, $\sum y_i=594.7$ so $\conj{y}=59.47$. I'm not really sure how to show my work for computing the other stuff, so here's some R code to compute S** (where * stands for either X or Y).
\begin{verbatim}
sss <- function(v1,v2) {
     v1_avg <- mean(v1); v2_avg <- mean(v2)
     sum((v1-v1_avg)*(v2-v2_avg))
}
\end{verbatim}

The slope is $SXY/SXX=0.582$, and the intercept is $\conj{y}-m\conj{x}=-36.87$. 

<insert plotted line over graph>
\ssn{3}
We have $\widehat{\sigma^2}=\frac{RSS}{8}$. Once again, here is a function to compute the RSS:
\begin{verbatim}
rss <- function(x, y, slope, intercept) {
    sum((y-(intercept+slope*x))^2)
}
\end{verbatim}
will calculate the RSS. Running this, we get $RSS=572.014$ so $\widehat{\sigma^2}=71.5$.
\end{document}
