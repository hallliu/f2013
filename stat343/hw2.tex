\documentclass{article}
\usepackage{geometry}
\usepackage[namelimits,sumlimits]{amsmath}
\usepackage{amssymb,amsfonts}
\usepackage{multicol}
\usepackage{graphicx}
\usepackage[cm]{fullpage}
\newcommand{\tab}{\hspace*{5em}}
\newcommand{\conj}{\overline}
\newcommand{\dd}{\partial}
\newcommand{\ep}{\epsilon}
\newcommand{\openm}{\begin{pmatrix}}
\newcommand{\closem}{\end{pmatrix}}
\DeclareMathOperator{\cov}{cov}
\newcommand{\nc}{\newcommand}
\newcommand{\rn}{\mathbb{R}}
\nc{\h}[1]{\widehat{#1}}
\nc{\ssn}[1]{\subsubsection*{#1}}
\nc{\vecx}{\mathbf{x}}
\nc{\vecy}{\mathbf{y}}
\nc{\vecv}{\mathbf{v}}
\begin{document}
Name: Hall Liu

Date: \today 
\vspace{1.5cm}

\subsection*{1}
\ssn{8.7.1}
We have $\h{U}=HU$. If $U=\openm1&0&\cdots&0\closem^T$, then $HU$ is just the first column of $H$, or $(h_{1j})$ for $1\leq j\leq n$.
\ssn{8.7.2}
The residuals are $U-\h{U}$. The first entry is $1-h_{11}$, and the other entries are $0-h_{1j}$ or $-h_{1j}$ for $2\leq j\leq n$.
\subsection*{2}
\ssn{a}
We have $\var(S_n)=\cov(S_n,S_n)$. Since covariance is a bilinear function, $\cov(\sum X_i, \sum X_i)=\sum\var(X_i)+\sum_{i\neq j}\cov(X_i,X_j)$. Since $\var(X_i)=\sigma^2$ for all $i$, this equals $n\sigma^2+\sigma^2\sum_{i\neq j}r_{ij}=n\sigma^2+n(n-1)\sigma^2r$.
\ssn{b}
Since $n$ is a scalar, $\var(S_n/n)=\frac{1}{n^2}\var(S_n)=\frac{\sigma^2}{n}+\frac{n-1}{n}\sigma^2r$
\ssn{c}
Ignore the $\sigma^2$ for the purposes of computing the ratio. The first term is $0.01$ when $n=100$, and the second term is $0.99\cdot r=0.0495$. This means that the contribution to the variance of $S_n$ from the codependence of the $X_i$ is almost $5$ times that of the contribution of the variance of the $X_i$, and so considering the $X_i$ independent based on $r<<1$ would result in a large mistake in the calculation of the variance of $S_n$.
\subsection*{3}

\begin{verbatim}
> summary(a)

Call:
lm(formula = gamble ~ sex + status + income + verbal, data = teengamb)

Residuals:
    Min      1Q  Median      3Q     Max 
-51.082 -11.320  -1.451   9.452  94.252 

Coefficients:
             Estimate Std. Error t value Pr(>|t|)    
(Intercept)  22.55565   17.19680   1.312   0.1968    
sex1        -22.11833    8.21111  -2.694   0.0101 *  
status        0.05223    0.28111   0.186   0.8535    
income        4.96198    1.02539   4.839 1.79e-05 ***
verbal       -2.95949    2.17215  -1.362   0.1803    
---
Signif. codes:  0 ‘***’ 0.001 ‘**’ 0.01 ‘*’ 0.05 ‘.’ 0.1 ‘ ’ 1

Residual standard error: 22.69 on 42 degrees of freedom
Multiple R-squared:  0.5267,	Adjusted R-squared:  0.4816 
F-statistic: 11.69 on 4 and 42 DF,  p-value: 1.815e-06
\end{verbatim}
\ssn{a}
Percentage of variation explained is the $R^2$ ratio, which is $0.5267$ here.
\ssn{b}
A call to \verb|sort(residuals(a))| gives $94.252$ as the largest positive residual at case $24$.
\ssn{c}
Mean is $8.173\times10^{-16}$ and median is $-1.451$.
\ssn{d}
This can be done with \verb|cor(residuals(a), predict(a))|, which gives $2.13\times10^{-17}$, which is practically zero after accounting for machine precision.
\ssn{e}
\verb|cor(residuals(a),teengamb$income)| gives $2.97\times10^{-17}$.
\ssn{f}
This is simply the value of the $\beta$ parameter for sex, since the difference between males and females has magnitude $1$ when treated numerically. Depending on which gender got assigned to $1$ and which one got assigned to $0$, either the prediction is greater for males by $22.11$ than females or vice versa.
\subsection*{4}
Note: This is a stripped-down version of a proof of Craig's theorem that I found online. 


\subsection*{5}

\end{document}
