\documentclass{article}
\usepackage{geometry}
\usepackage[namelimits,sumlimits]{amsmath}
\usepackage{amssymb,amsfonts}
\usepackage{multicol}
\usepackage{graphicx}
\usepackage[cm]{fullpage}
\newcommand{\tab}{\hspace*{5em}}
\newcommand{\conj}{\overline}
\newcommand{\dd}{\partial}
\newcommand{\ep}{\epsilon}
\newcommand{\openm}{\begin{pmatrix}}
\newcommand{\closem}{\end{pmatrix}}
\DeclareMathOperator{\cov}{cov}
\DeclareMathOperator{\rank}{rank}
\DeclareMathOperator{\im}{im}
\DeclareMathOperator{\Span}{span}
\DeclareMathOperator{\Null}{null}
\newcommand{\nc}{\newcommand}
\newcommand{\rn}{\mathbb{R}}
\nc{\cn}{\mathbb{C}}
\nc{\ssn}[1]{\subsubsection*{#1}}
\nc{\inner}[2]{\langle #1,#2\rangle}
\nc{\h}[1]{\widehat{#1}}
\nc{\tl}[1]{\widetilde{#1}}
\nc{\norm}[1]{\left\|{#1}\right\|}
\nc{\ta}{\theta}
\begin{document}

Name: Hall Liu

Date: \today 
\vspace{1.5cm}
\subsection*{1}
\ssn{a}
For the first coordinate, multiplying the matrix by the given vector gives $\alpha\sin(j\ta)-\sin(2j\ta)=\sin(j\ta)(\alpha-2\cos(j\ta))$. For the last coordinate, we get $-\sin((n-1)j\ta)+\alpha\sin(nj\ta)=\alpha\sin(nj\ta)-\sin(nj\ta)\cos(j\ta)+\cos(nj\ta)\sin(j\ta)$, but since $\cos(nj\ta)=\cos(j\ta)$ and $\sin(nj\ta)=-\sin(j\ta)$, we have that this is $\sin(nj\ta)(\alpha-2\cos(j\ta))$.

For all the other coordinates, 
\end{document}
