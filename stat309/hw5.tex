\documentclass{article}
\usepackage{geometry}
\usepackage[namelimits,sumlimits]{amsmath}
\usepackage{amssymb,amsfonts}
\usepackage{multicol}
\usepackage{graphicx}
\usepackage[cm]{fullpage}
\newcommand{\tab}{\hspace*{5em}}
\newcommand{\conj}{\overline}
\newcommand{\dd}{\partial}
\newcommand{\ep}{\epsilon}
\newcommand{\openm}{\begin{pmatrix}}
\newcommand{\closem}{\end{pmatrix}}
\DeclareMathOperator{\cov}{cov}
\DeclareMathOperator{\rank}{rank}
\DeclareMathOperator{\im}{im}
\DeclareMathOperator{\Span}{span}
\DeclareMathOperator{\Null}{null}
\newcommand{\nc}{\newcommand}
\newcommand{\rn}{\mathbb{R}}
\nc{\cn}{\mathbb{C}}
\nc{\ssn}[1]{\subsubsection*{#1}}
\nc{\inner}[2]{\langle #1,#2\rangle}
\nc{\h}[1]{\widehat{#1}}
\nc{\tl}[1]{\widetilde{#1}}
\nc{\norm}[1]{\left\|{#1}\right\|}
\nc{\ta}{\theta}
\nc{\vp}{\varphi}
\begin{document}

Name: Hall Liu

Date: \today 
\vspace{1.5cm}
\subsection*{1}
\ssn{a}
For the first coordinate, multiplying the matrix by the given vector gives $\alpha\sin(j\ta)-\sin(2j\ta)=\sin(j\ta)(\alpha-2\cos(j\ta))$. For the last coordinate, we get $-\sin((n-1)j\ta)+\alpha\sin(nj\ta)=\alpha\sin(nj\ta)-\sin(nj\ta)\cos(j\ta)+\cos(nj\ta)\sin(j\ta)$, but since $\cos(nj\ta)=\cos(j\ta)$ and $\sin(nj\ta)=-\sin(j\ta)$, we have that this is $\sin(nj\ta)(\alpha-2\cos(j\ta))$.

For all the other coordinates, we have that the $i$th coordinate is $\alpha\sin(ij\ta)-\sin((i-1)j\ta)-\sin((i+1)j\ta)$. Breaking up the last two terms with trig, we get 
\[\alpha\sin(ij\ta)-\sin(ij\ta)\cos(j\ta)+\sin(j\ta)\cos(ij\ta)-\sin(ij\ta)\cos(j\ta)-\sin(j\ta)\cos(ij\ta)=\sin(ij\ta)(\alpha-2\cos(j\ta)\]
as desired. Thus the given vector is an eigenvector for $T_\alpha$ with eigenvalue $\alpha-2\cos(j\ta)$. The requirement that $e^{j\ta}$ is conjugate to $e^{nj\ta}$ (imposed by the last coordinate) dictates that $j$ must run from $1..n$ and $\theta=\frac{\pi}{n+1}$. Thus we have a list of eigenvalues and eigenvectors.

As long as $\alpha>2\cos(\frac{\pi}{n+1})$, the matrix will be positive definite because cosine takes on its greatest value for low values of its argument.
\ssn{b}
The iteration matrix is a matrix with zeros everywhere except the superdiagonal and the subdiagonal, whose entries are all $\frac{1}{2}$. To compute the $2$-norm, note that it has the same form as $T_\alpha$, so its eigenvalues are $\cos(j\theta)$, and so the $2$-norm is the largest one, or $\cos(\theta)$. Since this is less than $1$, the iteration will converge. Now, since this is a symmetric matrix, it's orthogonally diagonalizable, so $\|B_\alpha^m\|=\|B_\alpha\|^m$, which gives us the rate of convergence as $\lim_{m\to\infty}\frac{-\log(\cos(\theta)^m)}{m}=-\log(\cos(\theta))$.

Gauss-Seidel: The matrix $T_\alpha$ satisfies property A after scaling by $\frac{1}{2}$, and this scaling term disappears once we form the iteration matrix. To find the eigenvalues of the scaled matrix, the discussion in the notes implies that the eigenvalues are either $0$ or squares of the singular values of the matrix with $-\frac{1}{2}$ on the diagonal and the subdiagonal. At the same time, the singular values of that matrix are the absolute values of the eigenvalues of the Jacobi iteration matrix, which are $\cos(j\theta)$. Thus, the greatest eigenvalue of the Gauss-Seidel iteration matrix is $\cos(\theta)^2$, which means that it converges because $\cos(\theta)<1$. In addition, this means that the rate of convergence is also squared, or $\log(\cos(\theta))^2$.

SOR converges for all $\omega\in(0,2)$ because $A$ is symmetric and positive definite.
\subsection*{2}
\ssn{a}
If $x^{(0)}=x$, then for all $k$, $x^{(k)}=x$, so $y^{(m)}$ can be written as $x\sum_{k=0}^m\alpha_k^{(m)}$. Since the $y^{(m)}$ are supposed to approximate the true solution $x$, it makes sense to constrain that sum to be $1$.
\ssn{b}
First, we note that $x^{(k)}=M^kx^{(0)}+\sum_{i=0}^{k-1}M^{i}b$. We have $x^{(0)}=e^{(0)}+x$, so substituting in gives $\sum_{i=0}^{k-1}M^{i}b+M^ke^{(0)}+M^kx$. Now, since $Mx=x-b$, we have $M^kx=x-\sum_{i=0}^{k-1}M^ib$, so substituting this back in gives $x^{(k)}=M^ke^{(0)}+x$, so $y^{(m)}=\sum_{k=0}^m\alpha_k^{(m)}(M^ke^{(0)}+x)=x+\sum_{k=0}^m\alpha_k^{(m)}M^ke^{(0)}$, and subtracting off the $x$ gives us $\ep^{(m)}$, which is now a polynomial of $M$ applied to $e^{(0)}$. The coefficients of the polynomial are the $\alpha^{(m)}$, which sum up to $1$.
\ssn{c}
Suppose that the limit isn't less than $1$. Then, for all $\ep>0$, there exists some arbitrarily large $m$ such that there is some associated $e_m^{(0)}$ of norm $1$ that satisfies $\|P_m(M)e_m^{(0)}\|_2\geq1-\ep$. Using the expression for $P_m(M)$ we found above, this is equivalent to 
\[\norm{\sum_{k=0}^m\alpha_k^{(m)}e_m^{(k)}}\geq1-\ep\implies\sum_{k=0}^m\alpha_k^{(m)}\|e_m^{(k)}\|\geq1-\ep\]
Let $m\to\infty$, and let $e^(0)$ be the limit of the $e_m^{(0)}$. We now suppose that the $e^{(k)}$ converge to zero. I can't find a good formal argument, but the idea is that if the norms of the $e^{(k)}$ drop off, then we can't get the sum above to be big enough since the $\alpha_k$ are constrained to sum to $1$, which results in a contradiction. %TODO: not true/handwavy
\ssn{d}
In this case, $y^{(m)}=\frac{1}{m+1}\sum_{k=0}^mx^{(k)}$. Fix $\ep>0$. Then there exists some $N$ where $\|x^{(k)}-x\|<\ep$ for all $k>N$. Split the above sum into the $k\leq N$ and $k>N$ parts. As $m\to\infty$, the contribution from the $k\leq N$ part vanishes, and the other part becomes close to $x$ due to the convergence of the $x^{(k)}$.

For a counterexample, work in $\rn^1$ and let $x^{(k)}=(-1)^k$, which is clearly not convergent. Then $y^{(k)}$ is the harmonic sequence interspersed with zeros, which converges to zero. 
\subsection*{3}
\ssn{a}
The given polynomial evaluated at $M$ is zero, since $\det(xI-M)$ is the characteristic polynomial of $M$. The denominator is the characteristic polynomial evaluated at $1$, and $1$ cannot be a root since the spectral radius of $M$ is strictly less than $1$.
\ssn{b}
Unitarily diagonalize $M$ as $Q\Lambda Q^T$. Then, $\|P(M)\|=\|P(\Lambda)\|$, which is a diagonal matrix. Its $2$-norm is simply the maximum of the $P(\lambda_i)$, which is in turn bounded above by the sup norm of $P$ on $[\lambda_\text{min},\lambda_\text{max}]$, so we seek to minimize this over $P$ subject to the degree constraint and that $P(1)=1$. 
\ssn{c}
We know that the Chebyshev polynomials have sup norm $1$ on the interval $[-1,1]$, and since we're declaring that $P(1)=1$, this is the best we can do. Further, we have that the first two Chebyshev polynomials evaluated at $1$ are $1$, and by the recurrence relation given in the notes, we can induct on $k$ and show that all the polynomials come out to be $1$ at $1$. 
\ssn{d}
At $x=1$, the given expression is obviously $1$. For other values of $x$, the numerator is the Chebyshev polynomial shifted so that its argument lies between $[-1,1]$, so the numerator is minimized. The denominator is there for normalization.
\ssn{e}
Suppose another solution exists and call it $f$. By the properties of Chebyshev polynomials, all the extrema of $P_m(x)$ take the values $\pm\|P_m(x)\|_\infty$. 
\subsection*{4}
\ssn{a}
Taking the new values of $\lambda$, the polynomials $P_m(x)$ now take the form $\frac{C_m(x/\rho)}{C_m(1/\rho)}$. Using the recurrence relation from the notes, we have
\[C_{m+1}(1/\rho)P_{m+1}(x)=C_{m+1}(x/\rho)=\frac{2x}{\rho}C_m(x/\rho)-C_{m-1}(x/\rho)=\frac{2x}{\rho}C_m(1/\rho)P_m(x/\rho)-C_{m-1}(1/\rho)P_{m-1}(x)\]
\ssn{b}
Use the result from (2b), which states that $y^{(m)}-x=P_m(M)e_0$. Then, we can write 
\begin{align*}
    y^{(m+1)}-x=P_{m+1}(M)e_0&=\frac{2M}{\rho}\frac{C_m(1/\rho)}{C_{m+1}(1/\rho)}P_m(M)e_0-\frac{C_{m-1}(1/\rho)}{C_{m+1}(1/\rho)}P_{m-1}(M)e_0\\
                             &=\omega_{m+1}M(y^{(m)}-x)-\frac{C_{m-1}(1/\rho)}{C_{m+1}(1/\rho)}(y^{(m-1)}-x)\\
                             &=\omega_{m+1}(My^{(m)}-x-b)-\frac{C_{m-1}(1/\rho)}{C_{m+1}(1/\rho)}(y^{(m-1)}-x)\\
\end{align*}
Now note that 
\[\frac{C_{m-1}(1/\rho)}{C_{m+1}(1/\rho)}-\omega_{m+1}=\frac{\rho C_{m-1}(1/\rho)-2C_m(1/\rho)}{\rho C_{m+1}(1/\rho)}=-1\]
by the recurrence relation on the $C_m$, so we can use this to combine the $x$s and cancel out, so then we get 
\[y^{(m+1)}=\omega_{m+1}(My^{(m)}-b)-\frac{C_{m-1}(1/\rho)}{C_{m+1}(1/\rho)}y^{(m-1)}\]
Using the $-1$ identity from above again to expand the coefficient of $y^{(m-1)}$ gives us what we want.
\ssn{c}
Write $M=Q\Lambda Q^T$ so that $\|P_m(M)\|_2=\|P_m(\Lambda)\|_2$. This is the maximum value of $P_m(x)=\frac{C_m(x/\rho)}{C_m(1/\rho)}$ on the spectrum of $M$. The largest value that $C_m(x/\rho)$ can take anywhere on $[-\rho,\rho]$ is $1$, and it so happens that $\rho$ is an eigenvalue of $M$ and it attains the value $1$ there. Therefore, $\|P_m(M)\|_2=\frac{1}{C_m(1/\rho)}$. Now, since $1/\rho>1$, we have $C_m(1/\rho)=\cosh(m\cosh^{-1}(1/\rho))$. Since hyperbolic cosine is a monotonically increasing function for argument $>0$, $C_m(1/\rho)$ is monotonically increasing in $m$, so $\|P_m(M)\|_2$ is monotonically decreasing in $m$.
\ssn{d}
We have $\cosh^{-1}(x)=\log(x+\sqrt{x^2-1})$, so $\exp(\cosh^{-1}(1/\rho))=\frac{1}{\rho}(1+\sqrt(1-\rho^2))$. On the other hand, 
\[(\omega-1)^{-1}=\frac{1+\sqrt{1-\rho^2}}{1-\sqrt{1-\rho^2}}=\frac{(1+\sqrt{1-\rho^2})^2}{\rho^2}=\left(\frac{1+\sqrt{1-\rho^2}}{\rho}\right)^2=e^{2\sigma}\]
and the desired equality follows.

Then, we have
\[\cosh(m\sigma)=\frac{e^{-2m\sigma}+1}{2e^{-m\sigma}}=\frac{(\omega-1)^m+1}{2(\omega-1)^{m/2}}\]
and inverting this gives us $\|P_m(M)\|_2$.
\ssn{e}
We have
\[\omega_m=\frac{2C_{m-1}(1/\rho)}{\rho C_{m}(1/\rho)}=\frac{2\|P_m(M)\|}{\rho\|P_{m-1}(M)\|}=\frac{2(1+(\omega-1)^{m-1})(\omega-1)^{1/2}}{\rho(1+(\omega-1)^m)}=\frac{2\sqrt{\omega-1}}{\rho}\frac{1+(\omega-1)^{m-1}}{1+(\omega-1)^m}\]
Since $\omega-1\in[0,1]$, this sequence is monotone decreasing. Further, as $m\to\infty$, the term on the right goes to $1$, so the limit is $\frac{2\sqrt{\omega-1}}{\rho}=\frac{1}{2}\sqrt{\frac{\omega-1}{\rho^2}}$. Solving the expression of $\omega$ for $\rho^2$, we get $-\frac{4}{\omega^2}+\frac{4}{\omega}$, so plugging this in gives
\[\frac{1}{2}\sqrt{\frac{\omega-1}{-\frac{4}{\omega^2}+\frac{4}{\omega}}}=\frac{1}{2}\sqrt{\frac{\omega^2(1-\omega)}{-4(\omega-1)}}=\omega\]
%TODO: not true, 1/2c_m(x)+1/2
\subsection*{5}
\ssn{a}
Here, $D$ is the identity, $L$ is $\openm0&0\\-M&0\closem$, and $U$ is $\openm0&-M\\0&0\closem$. Then, applying the SOR iteration gives
\[\openm x^{(k+1)}\\z^{(k+1)}\closem=\omega\left(\openm b\\b\closem-\openm0&0\\-M&0\closem\openm x^{(k+1)}\\z^{(k+1)}\closem-\openm0&-M\\0&0\closem\openm x^{(k)}\\z^{(k)}\closem\right)+(1-\omega)\openm x^{(k)}\\z^{(k)}\closem\]
Just looking at the top block gives us
\[x^{(k+1)}=\omega(b+Mz^{(k)})+(1-\omega)x^{(k)}=\omega(Mz^{(k)}-x^{(k)}+b)+x^{(k)}\]
as desired, and looking at the bottom block gives
\[z^{(k+1)}=\omega(b+x^{(k+1)})+(1-\omega)z^{(k)}=\omega(Mx^{(k+1)}-z^{(k)}+b)+z^{(k)}\]
\ssn{b}
Defining $y^{(m)}$ as given and using the above iterations, first assume $m=2k$. Then
\[y^{(m+1)}=z^{(k)}=\omega(Mx^{(k)}-z^{(k-1)}+b)+z^{(k-1)}=\omega(My^{(m)}-y^{(m-1)}+b)+y^{(m-1)}\]
and if $m=2k+1$, then
\[y^{(m+1)}=x^{(k+1)}=\omega(Mz^{(k)}-x^{(k)}+b)+x^{(k)}=\omega(My^{(m)}-y^{(m-1)}+b)+y^{(m-1)}\]
which is what we want.
\subsection*{6}
\ssn{a}
The gradient of a quadratic form $x^TAx$ is $A^Tx+Ax$, and the gradient of $b^Tx$ is $b$, so taken together the gradient of $\vp(x_k)$ is $A^Tx_k+Ax_k-2b=2Ax_k-2b=-2r_k$. Computing the second derivative, we have $2A$, which is positive definite. Therefore the stationary points minimize $\vp$.
\ssn{b}
If we want to minimize $\vp(x_k+\alpha p_k)$, we must have $\nabla\vp(x_k+\alpha p_k)=0$, or $2Ax_k+2\alpha Ap_k-2b=2\alpha Ap_k-2r_k=0$. Left-multiplying by $p_k^T$ gives $\alpha p_k^TAp_k-p_k^Tr_k$, or $\alpha=\frac{p_k^Tr_k}{p_k^TAp_k}$.
\ssn{c}
The expression for $\vp(x_{k+1})$ is the following. Subscripts are dropped because they're all subbed with $k$.
\[(x+\alpha p)^TA(x+\alpha p)-2b^T(x+\alpha p)=x^TAx+\alpha(p^TAx+x^TAp)+\alpha^2p^TAp-2b^Tx-2\alpha b^Tp\]
Note that $p^TAx=x^TAp$ because $A$ is symmetric. Subtracting off $\vp(x)=x^TAx-2b^Tx$ then gives
\[\frac{2(x^TAp)(r^Tp)}{p^TAp}+\frac{(r^Tp)^2}{p^TAp}-\frac{2(b^Tp)(r^Tp)}{p^TAp}\]
We have $r^T=b^T-x^TA$, so $x^TA=b^T-r^T$, and thus $x^TAp=b^Tp-r^Tp$. Then, expanding out the first term will cancel out the last term and subtract off twice the second term, so we get what we're looking for.
\ssn{d}
From the lectures, the iteration equation for steepest descent is $x^{(k+1)}=x^{(k)}+\alpha_kr^{(k)}$, which is the same as the one in (b) with $r^{(k)}$ in the place of $p_k$. The value of $\alpha_k$ is given in the lectures as $\frac{(r^{(k)T}r^{(k)}}{r^{(k)T}Ar^{(k)}}$, which is again the same as the $\alpha_k$ we derived in (b) with $r^{(k)}$ in the place of $p_k$.
\ssn{e}
$A$ is symmetric and positive definite, therefore orthogonally diagonalizable as $Q\Lambda Q^T$. Then 
\[\|P(A)x\|_A^2=x^TP(A)Q\Lambda Q^TP(A)x=x^TQP(\Lambda)Q^TQ\Lambda Q^TQP(\Lambda)Q^Tx=x^TQP(\Lambda)^2\Lambda Q^Tx=x^TP(\Lambda)^2Ax\]
We can bound this from above by replacing $P(\Lambda)$ with a matrix with $\max(P(\lambda_i))^2$ on the diagonals, thereby getting the upper bound $\max(P(\lambda_i))\|x\|_A^2$. Taking the square root on both sides gives the desired inequality.
\ssn{f}
We have $P_\alpha(A)(x_{k-1}-x_*)=x_{k-1}-x_*-\alpha Ax_{k-1}-\alpha Ax_*=x_{k-1}-x_*-\alpha Ax_{k-1}+\alpha b=x_{k-1}-x_*+\alpha r_{k-1}=x_k-x_*$, so the inequality follows by plugging into the bound in (e).
\ssn{g}
The given polynomial takes the value $0$ at $1/\alpha$. Since it's a line with a bend in it, the maximum value of $|1-\alpha t|$ must be attained at one of the endpoints, $t=\lambda_n$ or $t=\lambda_1$. For $1/\alpha<\frac{\lambda_1+\lambda_n}{2}\equiv \mu$, the maximum occurs at $t=\lambda_n$ and otherwise at $t=\lambda_1$, since the graph of $|1-\alpha t|$ is symmetric about $1/\alpha$. Then, the maximum as a function of $\alpha$ can be expressed as
\[
    g(\alpha)=\begin{cases}
        |1-\alpha\lambda_n|&\text{if }\alpha\geq\frac{1}{\mu}\\
        |1-\alpha\lambda_1|&\text{if }\alpha<\frac{1}{\mu}\\
    \end{cases}
\]
The minimum occurs precisely when $\alpha=\frac{1}{\mu}$, which gives the maximum value as $1-\frac{2\lambda_n}{\lambda_1+\lambda_n}=\frac{\lambda_1-\lambda_n}{\lambda_1+\lambda_n}$. Then, plugging this into the bound from (f) gives us what we want.
\end{document}
